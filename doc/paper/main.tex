\documentclass[12pt, letterpaper, preprint, comicneue]{aastex63}
%\usepackage[default]{comicneue} % comic sans font for editing
\usepackage[T1]{fontenc}
%%% This file is generated by the Makefile.
\newcommand{\giturl}{\url{https://github.com/changhoonhahn/provabgs}}
\newcommand{\githash}{6d00025}\newcommand{\gitdate}{2021-01-14}\newcommand{\gitauthor}{ChangHoon Hahn}

\usepackage{color}
\usepackage{amsmath}
\usepackage{natbib}
\usepackage{ctable}
\usepackage{bm}
\usepackage[normalem]{ulem} % Added by MS for \sout -> not required for final version
\usepackage{xspace}
\usepackage{csvsimple} 

\usepackage{graphicx}
\usepackage{pgfkeys, pgfsys, pgfcalendar}


% typesetting shih
\linespread{1.08} % close to 10/13 spacing
\setlength{\parindent}{1.08\baselineskip} % Bringhurst
\setlength{\parskip}{0ex}
\let\oldbibliography\thebibliography % killin' me.
\renewcommand{\thebibliography}[1]{%
  \oldbibliography{#1}%
  \setlength{\itemsep}{0pt}%
  \setlength{\parsep}{0pt}%
  \setlength{\parskip}{0pt}%
  \setlength{\bibsep}{0ex}
  \raggedright
}
\setlength{\footnotesep}{0ex} % seriously?

% citation alias

% math shih
\newcommand{\setof}[1]{\left\{{#1}\right\}}
\newcommand{\given}{\,|\,}
\newcommand{\lss}{{\small{LSS}}\xspace}

\newcommand{\Om}{\Omega_{\rm m}} 
\newcommand{\Ob}{\Omega_{\rm b}} 
\newcommand{\OL}{\Omega_\Lambda}
\newcommand{\smnu}{M_\nu}
\newcommand{\sig}{\sigma_8} 
\newcommand{\mmin}{M_{\rm min}}
\newcommand{\BOk}{\widehat{B}_0} 
\newcommand{\hmpc}{\,h/\mathrm{Mpc}}
\newcommand{\bfi}[1]{\textbf{\textit{#1}}}
\newcommand{\parti}[1]{\frac{\partial #1}{\partial \theta_i}}
\newcommand{\partj}[1]{\frac{\partial #1}{\partial \theta_j}}
\newcommand{\mpc}{{\rm Mpc}}
\newcommand{\eg}{\emph{e.g.}}
\newcommand{\ie}{\emph{i.e.}}

% cmds for this paper 
\newcommand{\gr}{g{-}r}
\newcommand{\fnuv}{FUV{-}NUV}
\newcommand{\sfr}{{\rm SFR}}
\newcommand{\ssfr}{{\rm SSFR}}
\newcommand{\mtaum}{m_{\tau,M_*}}
\newcommand{\mtaus}{m_{\tau,{\rm SSFR}}}
\newcommand{\ctau}{c_\tau}
\newcommand{\mdeltam}{m_{\delta,M_*}}
\newcommand{\mdeltas}{m_{\delta,{\rm SFR}}}
\newcommand{\cdelta}{c_\delta}
\newcommand{\eda}{EDA}


\newcommand{\specialcell}[2][c]{%
  \begin{tabular}[#1]{@{}c@{}}#2\end{tabular}}
% text shih
\newcommand{\foreign}[1]{\textsl{#1}}
\newcommand{\etal}{\foreign{et~al.}}
\newcommand{\opcit}{\foreign{Op.~cit.}}
\newcommand{\documentname}{\textsl{Article}}
\newcommand{\equationname}{equation}
\newcommand{\bitem}{\begin{itemize}}
\newcommand{\eitem}{\end{itemize}}
\newcommand{\beq}{\begin{equation}}
\newcommand{\eeq}{\end{equation}}

%% collaborating
\newcommand{\todo}[1]{\marginpar{\color{red}TODO}{\color{red}#1}}
\definecolor{orange}{rgb}{1,0.5,0}
\newcommand{\ch}[1]{{\color{orange}{\bf CH:} #1}}

\begin{document} \sloppy\sloppypar\frenchspacing 

%\title{Measuring Unbiased Star Formation Histories: Correcting Model Imposed Priors}
\title{Mitigating Model Priors in Galaxy Spectral Energy Distribution Fitting} 
\date{\texttt{DRAFT~---~\githash~---~\gitdate~---~NOT READY FOR DISTRIBUTION}}

\newcounter{affilcounter}
\author{ChangHoon Hahn}
\altaffiliation{changhoon.hahn@princeton.edu}
\affil{Department of Astrophysical Sciences, Princeton University, Peyton Hall, Princeton NJ 08544, USA} 

\begin{abstract}
    Models for galaxy star formation histories (SFHs), both parametric and
    non-parametric, impose strong priors on the physical properties of
    galaxies. These priors significantly bias galaxy stellar mass, star
    formation rate, and metalicities inferred from fitting galaxy spectral
    energy distributions (SED) and therefore impact all of the main summary
    statistics used to investigate galaxy populations (\eg~stellar mass function, star
    formation rate-density, star-forming sequence). In this work, we
    %demonstrate that the \cite{handley} method can correct for these biases by
    present a method that can correct for these biases by imposing uniform, or
    uninformative, priors, on the physical properties. The method imposes a
    maximum-entropy transformation on the probability distributions of the SED 
    model parameters to force the physical properties into any specified 
    distribution. We demonstrate, using simulated galaxy spectra constructed
    from the IllustrisTNG hydrodynamical simulation, that with this method we
    can accurately recover the input SFHs with SED modeling. Lastly, we use 
    the method to infer the SFHs of galaxies in a low-redshift, volume-complete
    sample from the Galaxy and Mass Assembly (GAMA) Survey. The cosmic star 
    formation rate-density we derive from the inferred SFHs are in good
    agreement with direct observations. 
\end{abstract}

\keywords{
keyword1 -- keyword2 -- keyword3
}

% --- intro ---  
\section{Introduction} \label{sec:intro} 

% --- methods ---  
\input{maxent}
% --- results ---  
\section{Results} \label{sec:results}
We are interested in estimating the SMF of BGS galaxies from their individual
marginalized posteriors, $p(M_* \given {\bfi X_i})$, derived using
PROVABGS~(Section~\ref{sec:provabgs}). 

we're going to do population inference in a hierarchical bayesian framework and
use a normalizing flow.
%We can infer the population hyperparameters, μ∆θ and σ∆θ , using a hierarchical Bayesian framework (e.g. Hogg et al. 2010; Foreman-Mackey et al. 2014; Baronchelli et al. 2020).

why? because it produces unbiased inference. 

why do we use normalizing flows? 

We follow the same approach as \cite{hahn2022} to estimate:
\begin{align}\label{eq:popinf}
p(\phi \given \{{\bfi X_i}\}) 
    =&~\frac{p(\phi)~p( \{{\bfi X_i}\} \given \phi)}{p(\{{\bfi X_i}\})}\\
    =&~\frac{p(\phi)}{p(\{{\bfi X_i}\})}\int p(\{{\bfi X_i}\} \given \{\theta_i\})~p(\{\theta_i\} \given \phi)~{\rm d}\{\theta_i\}.\\
    =&~\frac{p(\phi)}{p(\{{\bfi X_i}\})}\prod\limits_{i=1}^N\int p({\bfi X_i} \given \theta_i)~p(\theta_i \given \phi)~{\rm d}\theta_i\\
    =&~\frac{p(\phi)}{p(\{{\bfi X_i}\})}\prod\limits_{i=1}^N\int \frac{p(\theta_i \given {\bfi X_i})~p({\bfi X_i})}{p(\theta_i)}~p(\theta_i \given \phi)~{\rm d}\theta_i\\
    =&~p(\phi)\prod\limits_{i=1}^N\int \frac{p(\theta_i \given {\bfi X_i})~p(\theta_i \given \phi)}{p(\theta_i)}~{\rm d}\theta_i. 
\intertext{
    We estimate the integral using $S_i$ Monte Carlo samples from the
    individual posteriors $p(\theta_i \given {\bfi X_i})$: 
}
    \approx&~p(\phi)\prod\limits_{i=1}^N\frac{1}{S_i}\sum\limits_{j=1}^{S_i}
    \frac{p(\theta_{i,j} \given \phi)}{p(\theta_{i,j})}.
\end{align} 

BGS provides two samples: BGS Bright and Faint. 
Galaxies in BGS Bright are selected based on a $r < 19.5$ flux limit, while
the ones in BGS Faint are selected based on a fiber-magnitude and color limit
and $r < 20.0175$ flux limit. 
Since neither of these samples are volume-limited and complete as a function
of $M_*$, we must include the selection effect when estimating the SMF. 
We do this by including weights derived from $z^{\rm max}$, the maximum
redshift that galaxy $i$ could be placed and still be included in the BGS
samples. 
We derive $z^{\rm max}_{i,j}$ for every MCMC sample of 
$\theta_{i,j}\sim p(\theta \given {\bfi X_i})$ by redshifting the predicted
SED. 
We then derive $V^{\rm max}_{i,j}$, the comoving volume out to 
$z^{\rm max}_{i,j}$, and weights $w_{i,j} = 1/V^{\rm max}_{i,j}$. 
We modify Eq.~\ref{eq:popinf} to include $w_{i,j}$: 
\begin{align}
p(\phi \given \{{\bfi X_i}\}) 
    \approx&~\frac{p(\phi)}{\prod\limits_{i=1}^N p({\bfi X_i})^{w_i}} 
    \prod\limits_{i=1}^N \left(\int p({\bfi X_i} \given \theta_i)~p(\theta_i \given \phi)~{\rm d}\theta_i \right)^{w_i} \\ 
    \approx&~\frac{p(\phi)}{\prod\limits_{i=1}^N p({\bfi X_i})^{w_i}} 
    \prod\limits_{i=1}^N \left( \frac{1}{w_i} \sum\limits_{j=1}^{S_i} w_{i,j}
    \frac{p(\theta_{i,j} \given \phi)}{p(\theta_{i,j})} \right)^{w_i},
\end{align} 
where $w_i = \frac{1}{S_i} \sum\limits_{j=1}^{S_i} w_{i,j}$. 

In practice, we do not derive the full posterior 
$p(\phi \given \{{\bfi X_i}\})$. 
Instead we derive the maximum a posteriori (MAP) hyperparameter 
$\phi_{\rm MAP}$ that maximizes $p(\phi \given \{{\bfi X_i}\})$ or 
$\log p(\phi \given \{{\bfi X_i}\})$.
We expand, 
\begin{align}
\log p(\phi \given \{{\bfi X_i}\}) 
    \approx&~\log p(\phi) - 
    \sum\limits_{i=1}^N w_i \log w_i + 
    \sum\limits_{i=1}^N w_i \log \left(\sum\limits_{j=1}^{S_i} w_{i,j} \frac{p(\theta_{i,j} \given \phi)}{p(\theta_{i,j}} \right).
\end{align} 
Since the first two terms are constant, we derive $\phi_{\rm MAP}$ by
maximizing 
\begin{equation}
    \max_\phi \sum\limits_{i=1}^N w_i \log \left(\sum\limits_{j=1}^{S_i} w_{i,j} \frac{p(\theta_{i,j} \given \phi)}{p(\theta_{i,j}} \right).
\end{equation}
We use {\sc Adam} optimizer and determine the architecture of the normalizing
flow through experimentation.  


%\approx&~\log p(\phi) - 
%\log \prod\limits_{i=1}^N p({\bfi X_i})^{w_i} + 
%\log \prod\limits_{i=1}^N \left(\int p({\bfi X_i} \given \theta_i)~p(\theta_i \given \phi)~{\rm d}\theta_i \right)^{w_i} \\
%\approx&~\log p(\phi) - 
%\sum\limits_{i=1}^N w_i \log p({\bfi X_i}) + 
%\sum\limits_{i=1}^N w_i \log \left(\int p({\bfi X_i} \given \theta_i)~p(\theta_i \given \phi)~{\rm d}\theta_i \right) \\
%\approx&~\log p(\phi) + 
%\sum\limits_{i=1}^N w_i \log \left(\frac{1}{w_i} \sum\limits_{j=1}^{S_i} w_{i,j} \frac{p(\theta_{i,j} \given \phi)}{p(\theta_{i,j}} \right) \\

% --- summary ---  
\input{summary}

\section*{Acknowledgements}
It's a pleasure to thank
    Mariska Kriek, 
    Marius Millea, 
    Katherine Suess, 
    Jeremy Tinker, 
    Rita Tojeiro

\appendix

\bibliographystyle{mnras}
\bibliography{maxent} 
\end{document}
